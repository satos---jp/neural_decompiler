\documentclass[senior,final,11pt]{iscs-thesis}
%論文の種類とフォントサイズをオプションに
%\usepackage{graphicx}% 必要に応じて
%\usepackage{mysettings}% 自分用設定
%-------------------
\etitle{Neural Network based Decompiler for Machine Coding}
\jtitle{ニューラルネットを用いた機械語のための逆コンパイラ}
__NAMES__
\date{December 11, 2018}
%-------------------

\begin{document}
\begin{eabstract}
A decompiler is a tool for recovering a source code from compiled binary data.
There are various decompilers, but they often output a source code that is not intelligible to humans. 
In this thesis, we tried to apply machine translation techniques to generate human intelligible decompiled source codes. 
More specifically, we propose to use recurrent neural networks with attention, which are useful for machine translation. 
In experiments, we use source codes collected from open source projects and their binary data for training and evaluating the neural networks.
\end{eabstract}
\begin{jabstract}
逆コンパイラはコンパイル後のバイナリデータからソースコードを復元するためのツールである。
様々な逆コンパイラが存在するが、既存の逆コンパイラはしばしば人間にとって分かりにくいソースコードを出力する。
そこで本論文では、統計的機械翻訳の技術を逆コンパイラに用いた、解析者にとってわかりやすいソースコードを出力する逆コンパイラを提案する。
具体的には、機械翻訳において有用とされる注意機構付き再帰型ニューラルネットワークを用いる。
実験では、オープンソースプロジェクトから収集したソースコードとそのバイナリデータを用いてニューラルネットワークを学習し、その性能を検証した。
\end{jabstract}

\maketitle

\end{document}
